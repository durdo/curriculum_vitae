\documentclass[a4paper,10pt]{article}

%A Few Useful Packages
\usepackage{marvosym}
\usepackage{fontspec}                     %for loading fonts
\usepackage{xunicode,xltxtra,url,parskip} %other packages for formatting
\RequirePackage{color,graphicx}
\usepackage[usenames,dvipsnames]{xcolor}
\usepackage[big]{layaureo}                %better formatting of the A4 page
% an alternative to Layaureo can be ** \usepackage{fullpage} **
\usepackage{supertabular}                 %for Grades
\usepackage{titlesec}                     %custom \section

%Setup hyperref package, and colours for links
\usepackage{hyperref}
\definecolor{linkcolour}{rgb}{0,0.2,0.6}
\hypersetup{colorlinks,breaklinks,urlcolor=linkcolour, linkcolor=linkcolour}

%FONTS
\defaultfontfeatures{Mapping=tex-text}
\setmainfont[
SmallCapsFont = Fontin-SmallCaps.otf,
BoldFont = Fontin-Bold.otf,
ItalicFont = Fontin-Italic.otf
]
{Fontin.otf}
%%%

%CV Sections inspired by:
\titleformat{\section}{\Large\scshape\raggedright}{}{0em}{}[\titlerule]
\titlespacing{\section}{0pt}{3pt}{3pt}

%\hyphenation{im-pre-se}

%--------------------------------BEGIN DOCUMENT---------------------------------

\begin{document}

\pagestyle{empty} % non-numbered pages

\font\fb=''[cmr10]'' %for use with \LaTeX command

%------------------------------------TITLE--------------------------------------

\par
{\centering
    {\Huge \textsc{Eduardo Almeida dos Santos}
}\bigskip\par}

%-----------------------------------SECTIONS------------------------------------

%Section: Personal Data
\section{Contato}

\begin{tabular}{rl}
    %\textsc{Date of birth:}
    %& 13/02/1988 \\

    %\textsc{Address:}
    %& Rua Cascavel 3161, Bloco Lemans, Apt. 13, Curitiba-PR, Brasil \\

    \textsc{Celular:}
    & 41 99960-3141 \\

    \textsc{e-mail:}
    & \href{mailto:durdo.one@gmail.com}{durdo.one@gmail.com}
\end{tabular}

%-------------------------------------------------------------------------------

\section{Sumário Profissional}
    Profissional da área de engenharia de software com amplo conhecimento em
    desenvolvimento de aplicações de backend para servidores Linux. Atualmente
    desenvolvendo aplicações para telefonia em banda larga (VOIP) e
    infra-estrutura de sistema operacional Linux.

%-------------------------------------------------------------------------------

\section{Experiência Profissional}
    \begin{tabular}{r|p{11cm}}
        \emph{Atual}                & Analista de desenvolvimento em \textsc
                                     {Unify Communications},
                                     Curitiba \\

        \textsc{Setembro 2015}       &\emph
                                     {Soluções integradas para telecomunicações} \\
                                     &\footnotesize
                                     {
                                         Atua como um membro-chave no desenvolvimento
                                         das funcionalidades do Controlador de Borda
                                         de Sessão (SBC) da empresa. Mantenedor e
                                         principal contribuidor para fork do
                                         software-livre RTPProxy, utilizado para
                                         processamento de mídia em tempo-real e
                                         estabelecimento interativo de mídia.
                                         Contribuidor em diversas aplicações cujas
                                         funcionalidades variam desde protocos de
                                         de "baixo-nível" (SIP, RTP, DTLS, TLS,
                                         progração de sockets, etc.), passando por
                                         aplicações de infra-estrutura (redundância
                                         de servidores, ferramentas de análise de
                                         dados, etc) e eventualmente desenvolvendo
                                         frontend (em PHP/Javascript/HTML/CSS).
                                     } \\
                                     \multicolumn{2}{c}{} \\

        \emph{\textsc{Julho 2015}}   & Estagiáro/Especialista técnico em \textsc
                                     {Agres Sistemas Eletrônicos},
                                     Curitiba \\

        \textsc{Janeiro 2014}        &\emph
                                     {Agricutura de precisão} \\
                                     &\footnotesize
                                     {
                                         Desenvolvimento de funcionalidades,
                                         correção de problemas e criação/atualização
                                         de drivers de hardware proprietários em sistema
                                         operacional em tempo-real. Iniciou o processo
                                         de portabilidade do software de um sistema
                                         operacional em tempo real rodando em um ARM
                                         Cortex-M3 para um sistema Linux/Android rodando
                                         em kit de desenvolvimento iMx53. Implementação
                                         do protocolo de comunicação automotiva
                                         ISOBUS (ISO11783).
                                     } \\
                                     \multicolumn{2}{c}{} \\

        \emph{\textsc{Janeiro 2012}} & Estagiáro em \textsc
                                     {IBL Innovative Berlin Laser GmbH},
                                     Berlim \\

        \textsc{Agosto 2011}         &\emph
                                     {Lasers de estado sólido bombeados para
                                            indústria e aplicações científicas} \\
                                     &\footnotesize
                                     {
                                         Desenvolvimento de ferramentas de teste,
                                         aquisição de dados e controle de maquinário
                                         à laser.  Usando linguagem de programação
                                         gráfica (LabView) desenvolveu interface de
                                         usuário completa para controle de equipamento
                                         à laser.
                                     } \\
                                     \multicolumn{2}{c}{} \\
    \end{tabular}

%-------------------------------------------------------------------------------

%Section: Education
\section{Formação}
    \begin{tabular}{rl}
        \textsc{Abril} 2015 & Bacharel em: \\
                            &\textsc
                            {Engenharia Industrial Elétrica Ênfase
                                    Eletrônica/Telecomunicações }, \\

                            &\textbf
                            {Universidade Tecnológica
                                Federal do Paraná (UTFPR)},
                            Curitiba \\ \\

        \textsc{2011}       & Intercâmbio de um ano em: \\

                            & \textbf
                            {Beuth Hochschule für Technik Berlin},
                            Berlim \\ \\
    \end{tabular}

%-------------------------------------------------------------------------------

%Section: Languages
\section{Línguas}
    \begin{tabular}{rl}
        \textsc{Português:} &Nativo\\
        \textsc{Inglês:}    &Fluente\\
        \textsc{Alemão:}    &Básico\\
    \end{tabular}

\section{Habilidades}
    \begin{tabular}{rp{9cm}}
        Conhecimento básico:        & \textsc {php, javascript, css, html, java,
                                    pearl, C\#, assembly, VBA, protocolos de comunicação
                                    automotiva (ISOBUS, CanOpen, EtherCat), Android
                                    OS,} {\fb \LaTeX}\setmainfont
                                    [SmallCapsFont=Fontin-SmallCaps.otf]
                                    {Fontin.otf}. \\ \\

        Conhecimento intermediário: & \textsc{Microsoft Office, Shell script, Python,
                                    MatLab, LabView, diversos protocolos de rede.} \\ \\

        Conhecimento avançado:      & \textsc{GNU/Linux, Metodologia ágil, C, C++, git,
                                    redes, estabelecimento interativo de mídia (SIP,
                                    SDP, ICE, STUN, RTP, RTCP).}
    \end{tabular}

\section{Interesses e atividades}
    GNU/Linux, Vim, Tecnologia, Software-livre, Programação; \\
    Inteligência Artificial, Estruturas de dados, Algoritimos, Otimizações; \\
    Xadrez, Cubo de Rubik.

\end{document}
